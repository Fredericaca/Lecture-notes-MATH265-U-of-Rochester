\makeatletter
\ifx \nauthor\undefined
  \def\nauthor{Kaifeng Lu}
\else
\fi

\makeatletter
\ifx \email\undefined
  \def\email{\href{klu15@u.rochester.edu}{klu15 at u dot rochester dot edu}}
\else
\fi

\title{\coursenumber \, \coursename \, Class Notes}
\author{Based on lectures by Prof. \lecturer \\\small Notes taken by \nauthor\\\small Errata: \,\email}
\date{\term\ \year}

\usepackage{amsmath}
\usepackage{amsfonts}
\usepackage{amssymb}
\usepackage{amsthm}
\usepackage{caption}
\usepackage{enumitem}
\usepackage{fancyhdr}
\usepackage[margin=1in]{geometry}
\usepackage{listings}
\usepackage{multicol,multirow}
\usepackage[hidelinks]{hyperref}
\usepackage{parskip}
\usepackage{tikz}
\usepackage{wrapfig}
\usepackage{mathtools}
\usepackage{caption}
\usepackage{subcaption}

% Theorems-renew
\theoremstyle{definition}
\newtheorem*{aim}{Aim}
\newtheorem*{axiom}{Axiom}
\newtheorem*{claim}{Claim}
\newtheorem*{corollary}{Corollary}
\newtheorem*{conjecture}{Conjecture}
\newtheorem*{definition}{Definition}
\newtheorem*{example}{Example}
\newtheorem*{exercise}{Exercise}
\newtheorem*{fact}{Fact}
\newtheorem*{law}{Law}
\newtheorem*{lemma}{Lemma}
\newtheorem*{notation}{Notation}
\newtheorem*{proposition}{Proposition}
\newtheorem*{question}{Question}
\newtheorem*{theorem}{Theorem}
\newtheorem*{assumption}{Assumption}

\newtheorem*{remark}{Remark}
\newtheorem*{warning}{Warning}

% Special sets
\newcommand{\C}{\mathbb{C}}
\newcommand{\N}{\mathbb{N}}
\newcommand{\Q}{\mathbb{Q}}
\newcommand{\R}{\mathbb{R}}
\newcommand{\Z}{\mathbb{Z}}
\newcommand{\F}{\mathbb{F}}
\newcommand{\neighbor}[2]{V_{#1}{(#2)}}

% Brackets
\newcommand{\abs}[1]{\left\lvert #1\right\rvert}
\newcommand{\norm}[1]{\left\lVert #1\right\rVert}
\newcommand{\highlight}[1]{\textit{\textbf{\underbar{#1}}}}
\newcommand{\bra}{\langle}
\newcommand{\ket}{\rangle}

%Sequence
\newcommand{\seq}[2]{(#1_{#2})}
\newcommand{\seqfun}[2][n]{(f_{#1}(#2))}

%Abbv
\newcommand{\st}{\text{ s.t. }}
\newcommand{\limx}[2]{\lim_{x\rightarrow #1} {#2}}
\newcommand{\limn}[1]{\lim_{n\rightarrow\infty} {#1}}
\newcommand{\ds}{\displaystyle}

%RenewCommand
\renewcommand\qedsymbol{$\blacksquare$}

\hypersetup{
    colorlinks,
    citecolor=black,
    filecolor=black,
    linkcolor=black,
    urlcolor=black,
    pdfpagemode=FullScreen
}

%Figures
\newcommand{\sidefig}[2]{
  \begin{wrapfigure}{#1}{1.7\textwidth}
    \includegraphics[width=0.6\textwidth]{image/#2}
    \centering
  \end{wrapfigure}
}
\newcommand{\centerfig}[2][0.6]{
  \begin{figure}[h]
    \includegraphics[width=#1 \textwidth]{image/#2}
    \centering
  \end{figure}
}

